\section{Examples}
\subsection{Protocol ARP}

The purpose of this example is to demonstrate Address Resolution Protocol (ARP)
(Figure \ref{fig:ARP}).

% \begin{figure}[H]
% \centering
% \includegraphics[width=0.85\textwidth]{./images/ARP.png}
% \caption{IMUNES canvas presenting the network used for the ARP example}
% \label{fig:ARP}
% \end{figure}

\begin{enumerate}
\item	Start the simulation.
\item	Start Ethereal on eth0 of pc1. 
\item	Open capture, and check ``Update list of packets in real time''.
\item	Start capturing.
\item	Open shell window on pc1.
\item	Check the availability of the server1 (10.0.0.10). What is the response
of the ping command? Stop the ping after several probes (Ctrl-C). 
\item	Stop capturing.
\item	Look into the Ethereal window. What are the first two packets? What are
the source and destination addresses in those packets?
\item	The first packet is ARP request, sent by pc1 to the broadcast address,
requesting the MAC address of server1.
\item	The second packet is ARP response, sent by server1 to the pc1.
\item	Now, check the availability of server2 (10.0.8.10) while capturing the
traffic on pc1. Stop pinging and capturing after several probes.
\item	Look into the Ethereal window. What are the first two packets? What are
the source and destination addresses in those packets?
\item	The first packet is ARP request sent by pc1 to the broadcast address,
requesting the MAC address of router.
\item	The second packet is ARP reply sent by router to pc1. 
\item	Further work: monitor the traffic on server2 in both cases. Does
server2 see those ARP requests? Which ARP reqestst does it see? Why?
\end{enumerate}

\subsection{Ping example}

The purpose of this example is to demonstrate how ping works (Figure
\ref{fig:ping}).

% \begin{figure}[H]
% \centering
% \includegraphics[width=0.85\textwidth]{./images/Ping.png}
% \caption{IMUNES canvas presenting the network used for the Ping example}
% \label{fig:ping}
% \end{figure}

\begin{enumerate}
\item	Start the simulation.
\item	Start Ethereal on eth0 of pc1. 
\item	Open capture, and check ``Update list of packets in real time''.
\item	Start capturing.
\item	Open shell window on pc1.
\item	Check the availability of the server (10.0.8.10). What is the response
of the ping command? Stop the ping after several probes (Ctrl-C). 
\item	Stop capturing.
\item	Analyse the traffic from the ping:
\begin{itemize}
\item	How many packets were recorded?
\item	What are the first two packets?
\item	What packets are involved with each probe?
\item	What is Echo request?
\item	 What is Echo reply?
\end{itemize}
\item	Open the content of the ICMP header. Analyse it.
\item	Further work:
\begin{itemize}
\item	Repeat the same procedure with the increased packet size (1000 bytes,
2000 bytes). 
\item	Check 'man ping' for the info on how to use ping.  What happens? Does
the fragmentation occur? Analyse it.
\end{itemize}
\end{enumerate}

\subsection{Traceroute example}

The purpose of this example is to demonstrate how traceroute works (Figure
\ref{fig:traceroute}).

% \begin{figure}[H]
% \centering
% \includegraphics[width=0.85\textwidth]{./images/Traceroute.png}
% \caption{IMUNES canvas presenting the network used for the Traceroute example}
% \label{fig:traceroute}
% \end{figure}

\begin{enumerate}
\item	Start Start the simulation.
\item	Start Ethereal on eth0 of pc1. 
\item	Open capture, and check ``Update list of packets in real time''.
\item	Start capturing.
\item	Open shell window on pc1.
\item	Check the route to the server (10.0.8.10). Analyse the response from
the traceroute and compare it with the network. Check the IP addresses of
router interfaces involved in traffic routing.
\item	Open shell window on the server.
\item	Check the route back to the pc1 (10.0.0.21). Analyse the response from
the traceroute and compare it with the network. Check the IP addresses of
router interfaces involved in traffic routing. Compare it with the IP address
in the last traceroute.
\item	Stop capturing after approx. 10 seconds.
\item	Analyse the traffic from both traceroutes.
\item	The first trace should show all datagrams sent from pc1.
\item	The second trace should show only the datagrams that actually reached
pc1. Where are the others? Explain! 
\item	Further work:
\begin{itemize}
\item	Start the capture on one of the router's interfaces along the path.
\item	What traffic does that router "see" on the selected interface? Why?
\end{itemize}
\end{enumerate}

\subsection{Protocol DHCP}

The purpose of this example is to show how Dynamic Host Configuration Protocol
(DHCP) server works. It requires the understanding of Linux and shell (Figure
\ref{fig:DHCP}).

% \begin{figure}[H]
% \centering
% \includegraphics[width=0.85\textwidth]{./images/DHCP.png}
% \caption{IMUNES canvas presenting the network used for the DHCP example}
% \label{fig:DHCP}
% \end{figure}

\begin{enumerate}
 \item	Start the simulation.
 \item	Go to the console of the machine Imunes is run on and from the DHCP
directory (where this file is) start the script ``start\_dhcp'': \#
./start\_dhcp
 \item	This script sets up the clients and the server.
 \item	Go back to the Imunes GUI and start shell on the host pc3. Start the
Ethereal on pc3 and start capturing, with the ``Update traffic in real time''
option checked.
 \item	In the pc3 shell check the pc3 IP address: pc3\# ifconfig -a
 \item	Release the address: pc3\# dhclient -r
 \item  After a few seconds, request a new address for the eth0 interface: pc\#
dhclient eth0
 \item	Now, check the address: pc\# ifconfig -a
 \item	Stop capturing traffic. Check the packet trace.
 \item	First, check the DHCP release trace. What messages are transmitted?
 \item	Now, check how the client gets the IP address from the DHCP server.
Identify those packets!
  \begin{itemize}
    \item	DHCP Discover
      \begin{itemize}
	\item	sent by pc3 with the source IP 0.0.0.0 (remember, pc3 want's
new address, so uses 0.0.0.0 as the source, destination address is broadcast -
pc3 has no idea who the DHCP server is
      \end{itemize}
    \item	DHCP Offer
      \begin{itemize}
	\item	DHCP server sends its offer, offering IP address, DNS servers,
router, subnet mask
      \end{itemize}    
    \item	DHCP Request
      \begin{itemize}
	\item	pc3 sends broadcast requesting the offered address
      \end{itemize}    
    \item   DHCP ACK
      \begin{itemize}
	\item	DHCPserver approves the request for the address, defines the
lease time (10 minutes in this example) and sends default router and DNS
servers.
      \end{itemize}  
  \end{itemize}
\end{enumerate}


\subsection{Protocol RIP}

The purpose of the first Routing Information Protocol (RIP) example is to show
what happens in the "quiet" network - how routers exchange information about
their neighbours (Figure \ref{fig:RIP1}).

% \begin{figure}[H]
% \centering
% \includegraphics[width=0.85\textwidth]{./images/RIP1.png}
% \caption{IMUNES canvas presenting the network used for the first RIP example}
% \label{fig:RIP1}
% \end{figure}

\begin{enumerate}
 \item	Start the simulation.
 \item	Open Ethereal on serial port ser0 on router 3.
 \item	Open capture dialog. Check ``Update list of packets in real time''
option.
 \item	Start capturing.
 \item	After 30 seconds, you should have at least 2 UDP datagrams.
 \item	Stop capturing after approx. 1 min.
 \item	In the Ethereal, open one package. 
  \begin{itemize}
   \item	What are the source and destination address?
   \item	How often are the RIP packets transmitted?
  \end{itemize}
 \item	Show the contents of the first package and check the RIP data.
 \item	Each datagram containes routes collected by the sending router, with
the associate metric. Comment on the metric (number of hops).
 \item	Further work:
   \begin{itemize}
    \item	during the simulation, open the router3 console and shut down
the serial1 interface (ser1), simultaneously monitoring traffic on eg. serial
port 2 of router2. Watch what happens with the routes. 
    \item	what happens after 180 seconds?
    \item	bring the route back to life, while still monitoring the
traffic on router2. What happens?
   \end{itemize}
\end{enumerate}

The purpose of the next example is to show what happens after the router goes
down, and then goes back up (Figure \ref{fig:RIP2}).

% \begin{figure}[H]
% \centering
% \includegraphics[width=0.85\textwidth]{./images/RIP2.png}
% \caption{IMUNES canvas presenting the network used for the second RIP example}
% \label{fig:RIP2}
% \end{figure}

\begin{enumerate}
 \item	Start the simulation.
 \item	Open Ethereal on eth2 on router2 and start capturing the traffic with
the ``Update list of packets in real time'' checked.
 \item	Open router console on router2 (right-click, shell window, vtysh) and
type "show ip rip". This shows the routes that router2 has. Analyse the routing
table - available networks, next hops and metrics, as well as the time in the
rightmost column.
 \item	Now, stop router7 (right-click, stop). 
 \item	In router2 console watch what is happening, using "show ip rip" and
"show ip rip status". Look at the time of last updates. When Last Update
reaches 3 minutes, the "show ip rip" should show metric 16 for the network
10.0.4.10. Meanwhile, try to ping or traceroute the server (10.0.4.10).
 \item	Watch what happens after 3 minutes. The network reconfigures and
routing tables are refreshed when the new update packets arrive. Try traceroute
from pc to the server. Which path do the packets take?
 \item	Next, start the router7 again (right-click, start). Show what happens
with the routes. Try traceroute from pc to the server and comment on the route.
\end{enumerate}

\subsection{Protocol OSPF}

The purpose of the first Open Shortest Path First (OSPF) example is to show
what happens in the "quiet" network - how routers exchange information about
their neighbours (Figure \ref{fig:OSPF1}).

% \begin{figure}[H]
% \centering
% \includegraphics[width=0.85\textwidth]{./images/OSPF1.png}
% \caption{IMUNES canvas presenting the network used for the first OSPF example}
% \label{fig:OSPF1}
% \end{figure}

\begin{enumerate}
 \item	Start the simulation.
 \item	Open Ethereal on serial port ser0 on router 3.
 \item	Open capture dialog. Check ``Update list of packets in real time''
option.
 \item	Start capturing.
 \item	After 30 seconds, you should have at least 2 UDP datagrams.
 \item	Stop capturing after approx. 1 min.
 \item	In the Ethereal, open one package. 
  \begin{itemize}
   \item	What are the source and destination address?
   \item	How often are the OSPF packets transmitted?
  \end{itemize}
 \item	Show the contents of the first package and check the OSPF data.
 \item	Each datagram containes routes collected by the sending router, with
the associate metric. Comment on the metric (number of hops).
 \item	Further work:
   \begin{itemize}
    \item	during the simulation, open the router3 console and shut down
the serial1 interface (ser1), simultaneously monitoring traffic on eg. serial
port 2 of router2. Watch what happens with the routes. 
    \item	what happens after 180 seconds?
    \item	bring the route back to life, while still monitoring the
traffic on router2. What happens?
   \end{itemize}
\end{enumerate}

The purpose of the next example is to show what happens after the router goes
down, and then goes back up (Figure \ref{fig:OSPF2}).

% \begin{figure}[H]
% \centering
% \includegraphics[width=0.85\textwidth]{./images/OSPF2.png}
% \caption{IMUNES canvas presenting the network used for the second OSPF example}
% \label{fig:OSPF2}
% \end{figure}

\begin{enumerate}
 \item	Start the simulation.
 \item	Open Ethereal on eth2 on router2 and start capturing the traffic with
the ``Update list of packets in real time'' checked.
 \item	Open router console on router2 (right-click, shell window, vtysh) and
type "show ip ospf". This shows the routes that router2 has. Analyse the
routing table - available networks, next hops and metrics, as well as the time
in the rightmost column.
 \item	Now, stop router7 (right-click, stop). 
%  \item	In router2 console watch what is happening, using "show ip
%  ospf" and "show ip ospf status". Look at the time of last updates. When Last
%  Update reaches 3 minutes, the "show ip ospf" should show metric 16 for the
%  network 10.0.4.10. Meanwhile, try to ping or traceroute the server
%  (10.0.4.10).
 \item	Watch what happens after 3 minutes. The network reconfigures and
routing tables are refreshed when the new update packets arrive. Try traceroute
from pc to the server. Which path do the packets take?
 \item	Next, start the router7 again (right-click, start). Show what happens
with the routes. Try traceroute from pc to the server and comment on the route.
\end{enumerate}

Use the following router commands:

% \begin{itemize}
%  \item	show ip route - shows all routes
%  \item	show ip ospf route - shows ospf routes
%  \item	show ip ospf interface - show info about router's interfaces
%  \item	show ip ospf neighbor - show info about router's neighbours
% \end{itemize}
 
Notice the dead time in "show ip ospf neighbor" and check what happens after
the router7 is shut down. Look at how the dead time changes.

\subsection{Protocol DNS \& WWW example}

% \begin{figure}[H]
% \centering
% \includegraphics[width=0.85\textwidth]{./images/DNS-WWW.png}
% \caption{Connections between multiple IMUNES canvases for DNS \& WWW examples}
% \label{fig:DNS-WWW}
% \end{figure}
% 
% \begin{figure}[H]
% \centering
% \includegraphics[width=0.85\textwidth]{./images/DNS-Internet.png}
% \caption{IMUNES canvas presenting the Internet network}
% \label{fig:DNS-Internet}
% \end{figure}
% 
% \begin{figure}[H]
% \centering
% \includegraphics[width=0.85\textwidth]{./images/DNS-GEANT.png}
% \caption{IMUNES canvas presenting the GEANT network}
% \label{fig:DNS-GEANT}
% \end{figure}
% 
% \begin{figure}[H]
% \centering
% \includegraphics[width=0.85\textwidth]{./images/DNS-ARNES.png}
% \caption{IMUNES canvas presenting the ARNES network}
% \label{fig:DNS-ARNES}
% \end{figure}
% 
% \begin{figure}[H]
% \centering
% \includegraphics[width=0.85\textwidth]{./images/DNS-SRCE.png}
% \caption{IMUNES canvas presenting the SRCE network}
% \label{fig:DNS-SRCE}
% \end{figure}
% 
% \begin{figure}[H]
% \centering
% \includegraphics[width=0.85\textwidth]{./images/DNS-FER.png}
% \caption{IMUNES canvas presenting the FER network}
% \label{fig:DNS-FER}
% \end{figure}
% 
% \begin{figure}[H]
% \centering
% \includegraphics[width=0.85\textwidth]{./images/DNS-ZZT.png}
% \caption{IMUNES canvas presenting the ZZT network}
% \label{fig:DNS-ZZT}
% \end{figure}
% 
% \begin{figure}[H]
% \centering
% \includegraphics[width=0.85\textwidth]{./images/DNS-LABS.png}
% \caption{IMUNES canvas presenting the LABS network}
% \label{fig:DNS-LABS}
% \end{figure}

\begin{enumerate}
\item	Start imunes: \# imunes DNS.imn \&
\item	In GUI select: Experiment $\to$ Execute
\item	Cleanup the picture: select View $\to$ Show $\to$ 
\begin{itemize}
  \item  de-select "Interface Names"
  \item  de-select "Link Lables" 
\end{itemize}
\item	Start DNS servers: \# ./start\_dns
\item	Top level, root servers are:
\begin{itemize}
  \item   aRootServer: master (primary)
  \item   bRootServer, cRootServer: slaves 
\end{itemize}
\item	Other DNS servers:
\begin{itemize}
  \item dnsCom: master for .com.
\item	dnsOrg: master for .org.
\item	dnsSrce:  master for .hr.
\item	sunic.sunet.se:    slave for .hr.
\item	diana: master for .fer.hr.
\item	dnsTel: master for .tel.fer.hr.
\end{itemize}
\item	Right-click on PC named ``tito'' (in lower left corner of the ZZT
canvas) and start xterm on that host and start Ethereal on dnsTel. Start
capturing the traffic.
\item	\# nslookup pc.arnes.si
\item	Stop the capture after several seconds and analyse the traffic.
Illustrate the example from the slide - who is sending requests and who is
replying?
\item	To show how the web works, start web servers (make sure DNS is
working!).
\end{enumerate}

The purpose of this example is to demonstrate how WWW works.

\begin{enumerate}
\item	\# ./start\_www
\item	Web servers:
\begin{itemize}
\item	www.tel.fer.hr
\item	jagor.srce.hr
\end{itemize}
\item	Right-click on PC named ``tito'' (in lower left corner of the ZZT
canvas) to start xterm on that host and start Ethereal on the same pc: \# links
http://jagor.srce.hr.
\item	Follow the link to www.tel.fer.hr and quit the program. Stop the capture.
\item	Analyse the HTTP traffic. Use the "Follow HTTP stream" option to show
what goes on on the connection.
\end{enumerate}
