\section{Starting and terminating a simulation through CLI}
In addition to the \emph{File$\to$Open} option to open an .imn file and the
\emph{Experiment$\to$Execute} option used to initiate the virtual network
topology in the GUI, the simulation can be initiated through the command-line
interface (CLI) with the following command:

\texttt{\# imunes -b simple-network.imn}
 
Using IMUNES through GUI, the \emph{Experiment$\to$Terminate} option is used
for shutting down the simulation and cleaning up the virtual network topology
from the kernel. The CLI alternative for the latter is the following command:

\texttt{\# imunes -b -e experimentId}

The parameter \texttt{exeperimentId} represents the experiment identifier. In
order to get the experiment identifier you can use the \texttt{himage} command.
With \texttt{himage -l} you will get a list of identifiers of all started
experiments. 

% In Figure \ref{fig:***} you can see the output for the line:
%  
% \texttt{\# vimage -l}
% 
% As you can see, there are two experiment identifiers listed, one is *** and
% the other is ***. In the next Figure \ref{fig:***} you can see the output for
% the line:
% 
% \texttt{\# vimage -lr}
% 
% In this output, beside the experment identifiers, you also get the list of
% network layer nodes with the information to which experiment they belong.
% This can be useful if you have more than one experiment started on your PC
% and can't resolve which identifier belongs to which experiment. With the
% additional information, which nodes the experiment conatains, it can be
% easier to distinguish the experiments.   
