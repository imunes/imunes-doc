\section{Himage tool}
The jls and jexec commands can be impractical for creating scripts for
topologies since the output of the jls command is needed for starting the jexec
command. The jexec command can't take the hostname of the jail as an argument,
it can only take the jail name or jail id. The jail name is created from the
IMUNES experiment ID and the node identifier (not hostname). Every experiment
started in IMUNES has a different randomly generated experiment ID to enable
execution of multiple experiments at once.

IMUNES comes with the \texttt{himage} tool that enables the usage of hostnames
when starting commands in virtual nodes. The \texttt{himage} tool starts the
jexec command with the appropriate jail name so that the user doesn't have to
search for it. 

The \texttt{himage} command has the following options:
\begin{itemize}
\item \texttt{himage vi\_hostname command ...} - executes the command in the virtual
node with the specified hostname. If no command is specified, it starts an
interactive shell.
\item \texttt{himage -m vi\_hostname} - executes the command in the experiment
master jail. If no command is specified, it starts an interactive shell.
\item \texttt{himage -v vi\_hostname} - gets the jail node name of the virtual
node with the specified hostname
\item \texttt{himage -n vi\_hostname} - gets the node identifier for the
specified hostname.
\item \texttt{himage -e vi\_hostname} - gets the experiment ID in which the virtual
node with the specified hostname is running.
\item \texttt{himage -j/-i vi\_hostname} - gets the jail ID in which the virtual
node with the specified hostname is running.
\item \texttt{himage -d} - gets the hostname jail filesystem path on the host
machine.
\item \texttt{himage -l} - gets the experiment list with the experiment data.
\item \texttt{himage -ln} - gets the experiment list with node names.
\end{itemize}

\subsection{Examples}

Example of usage of the command himage on a node with the hostname "pc" to get a
list of running processes:

\texttt{\# himage pc ps ax} \hfill

If there are multiple experiments running and there are nodes with the same
hostnames in these experiments the \texttt{himage} command accepts the following
node specification where \texttt{vi\_hostname} is specified as
\texttt{hostname@eid}, where eid is the experiments' ID.

\texttt{\# himage hostname@eid command ...} \hfill

i.e:

\texttt{\# himage pc@i3d05a ps ax} \hfill

where \texttt{i3d05a} is the experiment ID of the running experiments. To find
out which experiments are running the \texttt{himage -l} command can be used as
well as \texttt{jls -h jid name host.hostname}.

Execute the \texttt{ifconfig} command the IMUNES node named \texttt{server}:

\texttt{\# himage server ifconfig}

\section{Hcp tool}

While the \texttt{himage} command is used for running programs inside virtual
nodes the \texttt{hcp} is used to copy files directly to the filesystem of
running mobile nodes, thus simplifying deployment of configuration files for
starting various services on virtual nodes.

Usage of the command \texttt{hcp}:

\texttt{hcp [cp\_command\_options] [vi\_hostname1:]filename
[vi\_hostname2:]filename} \hfill

The \texttt{hcp} command invokes the cp command with the specified options. If
the \texttt{vi\_hostname1} is specified the script copies a file from the
virtual node, otherwise it copies a file from the local folders. The second
\texttt{vi\_hostname2} specifies on which node the first file will be copied, if
not specified the file is copied to the local folders.

\texttt{vi\_hostname} is specified in the same way as in the \texttt{himage}
command, \texttt{hostname} or \texttt{hostname@eid}.

\subsection{Examples}

Copy file dhcpd.conf from a local folder to the virtual node DHCP:

\texttt{\# hcp dhcpd.conf DHCP:/usr/local/etc/} \hfill

Copy file message.txt from the virtual node PC to a local folder:

\texttt{\# hcp PC:/root/message.txt .} \hfill

Copy file index.html from the virtual node HOST to the virtual node HTTP:

\texttt{\# hcp HOST:/usr/local/www/data/index.html HTTP:/usr/local/www/data/} \hfill

\section{Example (himage and hcp)}

This is an example of starting an DHCP server through a script with the
provided configuration file dhcpd.conf.

Kill the server if it's started:

\texttt{\# himage DHCP killall -9 dhcpd} \hfill

Copy the configuration file:

\texttt{\# hcp dhcpd.conf DHCP:/usr/local/etc/} \hfill

Create the \texttt{.leases} file defined in \texttt{dhcpd.conf}:

\texttt{\# himage DHCP touch /var/db/dhcpd.leases} \hfill

Start the sever with the copied configuration file:

\texttt{\# himage DHCP dhcpd -cf /usr/local/etc/dhcpd.conf} \hfill

\section{Vlink tool}

The \texttt{vlink} tool is used to change link parameters. The following link parameters are available:
\begin{itemize}
	\item bandwidth (bps, bits-per-second)
	\item bit-error rate, BER (number of bits in which one error will occur)
	\item delay (microseconds)
	\item packet duplication (\%, percentage of packets that will be duplicated)
\end{itemize}

A link is identified by the endpoint node names. The link between pc1 and pc2 is identified by \emph{pc1:pc2} or \emph{pc2:pc1}.

Usage of the command \texttt{vlink}:

\texttt{vlink [options] link\_name[@eid]} \hfill

\subsection{Examples}

Setting the bandwidth to 10 Mb/s with a delay of 30 ms to the link connecting \emph{router1} and \emph{pc1}:

\texttt{\# vlink -bw 10000000 -dly 30000 router1:pc1}

Generate an error on one bit in a million:

\texttt{\# vlink -BER 1000000 router1:pc1}

Set packet duplication to 20%:

\texttt{\# vlink -dup 20 router1:pc1}

Modifying a link in a specific experiment (i.e. Experiment ID = i56ad1):

\texttt{\# vlink -dup 20 router1:pc1@i56ad1}

To reset the link settings to the default values the -r flag is used:

\texttt{\# vlink -r router1:pc1@i56ad1}

Check all \texttt{vlink} arguments with:

\texttt{\# vlink -h}
