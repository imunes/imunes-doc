\chapter{Introduction}
IMUNES is an Integrated Multiprotocol Network Emulator / Simulator of IP based
networks. Virtual nodes in IMUNES are multiple network stack instances that are
formed through special FreeBSD kernel modifications and Docker containters on
Linux. Virtual nodes can be linked either with other virtual nodes or with the
physical network interface through simulated links. All virtual nodes share a
single place for their application binaries and libraries. The main strengths
of this tool are high scalability, performance and fidelity.

\section{Document overview}
This document is intended to be a manual for users that are getting started
with IMUNES, likewise for the ones that want to know more about its advanced
features.

This manual is divided into three main parts: User Interface Layout, Quick
Intro and Advanced Usage. 

The first part, User Interface Layout, gives detailed description of IMUNES
graphical user interface. 

The second part, Quick Intro, is intended to prepare beginners to get a working
network simulation in a short time. It gives detailed explanations for
building, configuring and simulating a simple network. At the end it gives
instructions related to IMUNES configuration files.

The third part, Advanced Usage, gives instructions for extending the network
topology built in the first section. It also explains the usage of additional
tools and configuration possibilities. It proceeds with features for
customizing look, such as annotations, background image and icon size. At the
end, it gives instructions related to event scheduling, starting and
terminating simulation through command-line interface and himage, hcp and vlink
commands.
