\section{Vimage - FreeBSD kernel command}
The FreeBSD jail mechanism allows partitioning of a FreeBSD-based computer
system into several independent smaller systems called jails. This mechanism
enables creation of a safe environment, separate from the rest of the system.
Processes created inside a jail are limited within that jail environment. Each
jail is a virtual environment running on the host machine, having its own file
system, processes, set of users, networking subsystem of the FreeBSD kernel and
a few other things.     

The vimage utility is an alternative user interface for controlling virtual
network stacks in FreeBSD, aimed primarily at supporting legacy applications
which are not yet converted to using \emph{jail, jexecn}, and \emph{jls}
command. So the vimage command in its "inside" is composed of or uses these
commands. 

A virtual image or vimage is a jail with its own independent network stack
instance.  Every process, socket and network interface present in the system is
always attached to one, and only one, virtual network stack instance (vnet).
During system bootup sequence a default vnet is created to which all the
configured interfaces and user processes are initially attached.  Assuming that
enough system resources are available, a user with sufficient privileges can
create and manage a hierarchy of subordinated virtual images.  The vimage
command allows for creation, deletion and monitoring of virtual images, as well
as for execution of arbitrary processes in a targeted virtual image.

\texttt{vimage vname [command ...]} \hfill

If invoked with no modifiers, the vimage command spawns a new interactive shell
in virtual image \texttt{vname}.  If optional additional arguments following
\texttt{vname} are provided, the first of those will be executed in place of
the interactive shell, and the rest of the arguments will be passed as
arguments to the executed command.

\texttt{vimage -l [-rvj] [vname]} \hfill
 
List the properties and statistics for virtual images one level below the
current one in the hierarchy. If an optional argument \texttt{vname} is
provided, only the information regarding the target virtual image
\texttt{vname} is displayed.  With the optional \texttt{vimage -r} switch
enabled the list will include all virtual images below the current level in the
vimage hierarchy.  Enabling the optional \texttt{vimage -v} or \texttt{vimage
-j} switches results in a more detailed output.

\subsection{Examples}

Execute the \texttt{ifconfig} command in the virtual image \texttt{v1}:

\texttt{\# vimage v1 ifconfig}

Show the status information for virtual image \texttt{v1}:

\texttt{\# vimage -lv v1}

For more details see the \emph{vimage(8)} man page.

% Virtual nodes corespond to vimages. Each vimage has its
% \begin{itemize}
%   \item fully private/independent interface list,
%   \item loopback interface instance,
%   \item indepentendt IPv4/IPv6 address space,
%   \item private routing table,
%   \item port/socket space,
%   \item netgraph space,
%   \item process group (technically, process grouping independent from stack virtualization).
% \end{itemize}
% 
% To list all vimages under current position in the hierarchy, including the current one, use the following command:\\
% 
% \texttt{\# vimage -l}\\
% 
% The \emph{default} vimage '.' is always present even if no experiment is started. Other commands:
% \begin{itemize}
%   \item \emph{vimage (no arguments)} - displays current vimage name
%   \item \emph{vimage -lr} 
%   \item \emph{vimage name command} - executes a command in context of vimage name
%   \item \emph{vimage name} - spawns an interactive shell in context of the target vimage
% \end{itemize}
