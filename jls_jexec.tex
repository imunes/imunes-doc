\section{Managing virtual nodes (jails) - jls, jexec}
The FreeBSD jail mechanism allows partitioning of a FreeBSD-based computer
system into several independent smaller systems called jails. This mechanism
enables creation of a safe environment, separate from the rest of the system.
Processes created inside a jail are limited within that jail environment. Each
jail is a virtual environment running on the host machine, having its own file
system, processes, set of users, networking subsystem of the FreeBSD kernel and
a few other things.     

Two main commands exist in FreeBSD for manging and configuring previously
created jails:
\begin{itemize}
\item \texttt{jexec} - executes a command inside an existing jail
\item \texttt{jls} - lists jails.
\end{itemize}

A virtual image or vimage is a jail with its own independent network stack
instance.  Every process, socket and network interface present in the system is
always attached to one, and only one, virtual network stack instance (vnet).
During system bootup sequence a default vnet is created to which all the
configured interfaces and user processes are initially attached. 

The \texttt{jexec} command allows for execution of arbitrary processes in a
targeted virtual image.

\texttt{jexec jname command ...} \hfill

The jexec command starts the selected command and it's arguments in the jail
\texttt{jname}.

To find out the names of started jails the jls command is used: 

\texttt{jls [-hnqsv] [parameter ...]} \hfill
 
Since the default jls command doesn't list names of jails a better output is
provided using the command:

\texttt{jls -h jid name host.hostname} \hfill

Also, the command \texttt{jls -v} gives a more detailed output.

\subsection{Examples}

Execute the \texttt{ifconfig} command in the jail with the jail name
\texttt{n1}:

\texttt{\# jexec n1 ifconfig} \hfill

Execute the \texttt{csh} command in the IMUNES virtual node named \texttt{host1}:

First we need to find out the jail ID or jail name to execute the wanted
command:

\texttt{\# jls -h jid name host.hostname | grep host1} \hfill

The first parameter output is the jail ID, the second is the jail name and the
last is the hostname. To execute the command you need the jail ID or jail name:

\texttt{\# jexec \emph{jid} csh} \hfill

\texttt{\# jexec \emph{jname} csh} \hfill
